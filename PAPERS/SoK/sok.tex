\documentclass[12pt]{article}
\usepackage{enumitem}
\usepackage{amssymb}
\usepackage{amsmath}
\usepackage{geometry}
\geometry{margin=1in}

\title{Systemisation of Knowledge: Digital Liquidity}
%  \large Planning Document for Thesis Chapter \& Journal Article}
%\author{(Planning Outline -- No References Yet)}
%\date{}

\begin{document}
\maketitle

%\tableofcontents

%%%%%%%%%%%%%%%%%%%%%%%%%%%%%%%%%%%%%%%%%%%%%%%%%%%%%%%%%%%%
\section{Purpose and Positioning}
%%%%%%%%%%%%%%%%%%%%%%%%%%%%%%%%%%%%%%%%%%%%%%%%%%%%%%%%%%%%

\subsection{High-Level Aims}

A unified, event-centric framework for
\emph{digital liquidity} across both legacy and tokenised infrastructures.

%Core aim:
\begin{itemize}[noitemsep]
  \item Provide a \textbf{minimal, machine-checkable event model} (BSTE) for balance-state transitions.
  \item Define a \textbf{Digital Liquidity Stack} that captures money nature, ledger technology, clearing, settlement, finality, credit, encumbrance, and interoperability.
  \item Use this to systematically classify payment systems, FMIs, and tokenised systems (RTGS, ACH/DNS, card schemes, CLS, CCP cash, stablecoins, CBDCs, bridges, AMMs, unified ledgers).
  \item Surface \textbf{patterns and failure modes} in digital liquidity management.
\end{itemize}

\subsection{Meaning of ``Digital Liquidity''}

Liqudity has multiple meanings:

\begin{enumerate}[label=(\alph*), noitemsep]
  \item \textbf{Market liquidity}: order-book depth, bid--ask spreads, price impact, AMM slippage.
  \item \textbf{Funding and settlement liquidity}: ability of institutions and infrastructures to obtain and deploy cash/collateral to meet obligations on time.
\end{enumerate}

In this work:

\begin{quote}
\textbf{Digital liquidity} means the \emph{capacity of digital infrastructures and participants to execute balance-state transitions} (our BSTEs)---to make obligations good in the right asset, on the right ledger, at the right time.
\end{quote}

We \emph{explicitly} do \textbf{not} attempt to survey or systematise market microstructure (order books, AMM curve design, price impact).
Those questions are treated as upper-layer phenomena and will be the subject of subsequent work.

\subsection{Intended Contributions}

The chapter / article delivers:

\begin{enumerate}[label=(C\arabic*), leftmargin=2cm]
  \item \textbf{BSTE model}: a minimal, compositional event representation for digital value movements.
  \item \textbf{Digital Liquidity Stack}: a 10-dimension core classification plus extended attributes to locate any system in a design space.
  \item \textbf{Mechanism taxonomy}: a systematic description of holds, locks, collateralisation, credit, queues, netting, LSM, PvP/DvP/PoP, channels, and bridges as compositions of BSTEs.
  \item \textbf{Comparative mapping}: worked classifications of representative legacy rails and tokenised systems.
  \item \textbf{Research agenda}: a bridge from this plumbing-level SoK to (i) formal optimisation (Paper~2) and (ii) empirical market stylised facts (Paper~3).
\end{enumerate}

%%%%%%%%%%%%%%%%%%%%%%%%%%%%%%%%%%%%%%%%%%%%%%%%%%%%%%%%%%%%
% \section{Position in the Thesis and Paper Arc}
% %%%%%%%%%%%%%%%%%%%%%%%%%%%%%%%%%%%%%%%%%%%%%%%%%%%%%%%%%%%%

% \subsection{Role as Opening Thesis Chapter}

% Within the PhD thesis, this chapter:

% \begin{itemize}[noitemsep]
%   \item Establishes the \textbf{core vocabulary and IR} (intermediate representation) for all later chapters.
%   \item Sets out the systematisation of \emph{instruments} and \emph{infrastructures} at the settlement layer.
%   \item Frames the thesis-wide research questions around digital liquidity, not just ``blockchain vs banks''.
% \end{itemize}

% \subsection{Relationship to the Other Papers}

% \paragraph{Paper 2 -- Liquidity Calculus.}
% Takes BSTE + stack and develops:

% \begin{itemize}[noitemsep]
%   \item graph-theoretic models of multi-ledger liquidity,
%   \item scheduling and routing problems for BSTEs,
%   \item worst-case encumbrance cost measures and optimisation algorithms.
% \end{itemize}

% \paragraph{Paper 3 -- Stylised Facts of Tokenised Markets.}
% Uses BSTE encoding and stack classifications to:

% \begin{itemize}[noitemsep]
%   \item measure settlement latencies, failure modes, and flows,
%   \item study collateralisation patterns and cross-venue fragmentation,
%   \item connect observed market behaviour to underlying plumbing constraints.
% \end{itemize}

% \paragraph{Paper 4 -- Collateral Instruments as Settlement Assets.}
% Builds on the SoK to discuss:

% \begin{itemize}[noitemsep]
%   \item repo-native money, tokenised collateral as settlement assets,
%   \item governance and legal finality aspects encoded in the stack.
% \end{itemize}

%%%%%%%%%%%%%%%%%%%%%%%%%%%%%%%%%%%%%%%%%%%%%%%%%%%%%%%%%%%%
\section{Scope of Background Review}
%%%%%%%%%%%%%%%%%%%%%%%%%%%%%%%%%%%%%%%%%%%%%%%%%%%%%%%%%%%%

The following topics are in scope: %section will \emph{not} be fully fleshed in the planning doc, but will outline what needs to be covered:

\begin{itemize}
  \item Classical payment systems and FMI literature:
    \begin{itemize}[noitemsep]
      \item RTGS design, LSM and gridlock-resolution algorithms.
      \item ACH/DNS systems, netting, risk management.
      \item CCPs and CLS (PvP systems), liquidity implications.
    \end{itemize}
  \item CBDC and unified-ledger architecture papers.
  \item Stablecoins, tokenised deposits, and tokenised collateral networks.
  \item DeFi systems: AMMs, lending protocols, cross-chain bridges.
  \item Existing ``stacks'' or taxonomies (e.g., generic blockchain stacks, CBDC design taxonomies).
  \item SoK methodology references (how SoK papers are typically structured).
\end{itemize}

The focus is \textbf{not} on exhaustive survey, but on situating this work among:
\begin{itemize}[noitemsep]
  \item payment/FMI engineering,
  \item tokenisation / DLT infrastructure,
  \item SoK literature on distributed systems and crypto.
\end{itemize}

%%%%%%%%%%%%%%%%%%%%%%%%%%%%%%%%%%%%%%%%%%%%%%%%%%%%%%%%%%%%
\section{BSTE: Balance-State Transition Event}
%%%%%%%%%%%%%%%%%%%%%%%%%%%%%%%%%%%%%%%%%%%%%%%%%%%%%%%%%%%%

\subsection{Economic Owner Abstraction}

Introduce a conceptual mapping:
\[
  \mathrm{econ\_owner}(account) \to \text{beneficial owner (bank, customer, CCP, etc.)} .
\]

This allows us to distinguish:

\begin{itemize}[noitemsep]
  \item movements that change who owns the claim,
  \item movements that only change how an owner’s claim is encumbered.
\end{itemize}

\subsection{Primitive Event Types}

\subsubsection*{Primitive P1: OWNERSHIP\_TRANSFER}

\begin{itemize}
  \item Economic owner multiset changes.
  \item Supply $S$ on the ledger is unchanged.
  \item Examples:
    \begin{itemize}[noitemsep]
      \item RTGS credit from bank A to bank B.
      \item On-chain ERC20 transfer.
      \item AMM swap legs where users receive tokens and AMM pool balances change.
    \end{itemize}
\end{itemize}

\subsubsection*{Primitive P2: ENCUMBRANCE\_ADJUST}

\begin{itemize}
  \item Economic owner set \emph{unchanged}, but claims move between:
    \begin{itemize}[noitemsep]
      \item free balance (\texttt{available}),
      \item encumbered buckets (holds, collateral, escrow, channels, etc.).
    \end{itemize}
  \item Constraint: $\mathrm{econ\_owner}(\mathrm{src\_account}) = \mathrm{econ\_owner}(\mathrm{dst\_account})$.
  \item Examples:
    \begin{itemize}[noitemsep]
      \item Card pre-authorisation: available $\to$ reservation.
      \item Central bank collateralisation: available $\to$ collateral.
      \item HTLC lock: available $\to$ channel or bridge\_lock.
      \item Release of a hold: reservation $\to$ available.
    \end{itemize}
\end{itemize}

\subsubsection*{Primitive P3: SUPPLY\_ADJUST}

\begin{itemize}
  \item Net change in recognised supply $S$.
  \item Exactly one of \verb|src_account|, \verb|dst_account| equals
        \verb|EXTERNAL_SOURCE| or \verb|EXTERNAL_SINK|.
  \item Examples:
    \begin{itemize}[noitemsep]
      \item Central bank monetary operations in reserves.
      \item Stablecoin mint/burn.
      \item Protocol base-fee burn.
    \end{itemize}
\end{itemize}

\subsection{Canonical BSTE Schema}

We adopt the following design choices:

\begin{itemize}[noitemsep]
  \item \verb|amount| is strictly positive.
  \item No structural \texttt{NULL}; use explicit sentinels.
  \item Encumbrance is derived from bucket types; no \verb|lock_delta|.
\end{itemize}

\subsubsection*{Identity and Ordering}

\begin{enumerate}[label=\arabic*.]
  \item \verb|event_id| : unique identifier.
  \item \verb|ledger_id| : authoritative balance-record; may be CB-RTGS, bank core, scheme settlement file, or DLT state machine.
  \item \verb|asset_code| : identifies the asset.
  \item \verb|op_kind| : \{OWNERSHIP\_TRANSFER, ENCUMBRANCE\_ADJUST, SUPPLY\_ADJUST\}.
  \item \verb|t_occurred| : logical posting time.
  \item \verb|event_seq| : per-ledger total order index.
\end{enumerate}

\subsubsection*{Accounts and Balance Types}

\begin{enumerate}[label=\arabic*.]
  \setcounter{enumi}{6}
  \item \verb|src_account| : string, or \verb|EXTERNAL_SOURCE|.
  \item \verb|src_balance_type| : \{available, reservation, collateral, escrow, channel, internal\_pending, bridge\_lock, other\}.
  \item \verb|dst_account| : string, or \verb|EXTERNAL_SINK|.
  \item \verb|dst_balance_type| : same enum.
\end{enumerate}

Encumbered buckets are those with \verb|balance_type| $\neq$ \texttt{available} (but we may specify the exact list).

\subsubsection*{Supply and Economic Role}

\begin{enumerate}[label=\arabic*.]
  \setcounter{enumi}{10}
  \item \verb|supply_delta| : signed change to supply on the ledger.
  \item \verb|economic_role| : \{principal, fee, tax, interest, margin, collateral\_movement, other\}.
  \item \verb|tax_subtype| : optional detail for \texttt{economic\_role = tax}.
\end{enumerate}

All fees/taxes are represented as ordinary BSTEs with appropriate \verb|economic_role|.

\subsubsection*{Grouping and Atomic Sets}

\begin{enumerate}[label=\arabic*.]
  \setcounter{enumi}{13}
  \item \verb|group_id| : business group (FX trade, batch, clearing cycle).
  \item \verb|link_id| : groups events into an atomic or quasi-atomic set.
\end{enumerate}

Atomic semantics are kept in a separate table \texttt{atomic\_sets}, keyed by \verb|link_id|, with fields:

\begin{itemize}[noitemsep]
  \item \verb|atomic_pattern| : none, PvP, DvP, PoP.
  \item \verb|atomic_mechanism| : single\_ledger\_tx, central\_novation, htlc, escrow\_agent, optimistic\_with\_fraud\_proof, trusted\_coordinator.
  \item \verb|atomic_params| : JSON (timeout heights, hashlocks, etc.).
  \item \verb|fx_rate| and \verb|price_reference| : optional for cross-asset sets.
\end{itemize}

\subsubsection*{Evidence, Expiry, Notes}

\begin{enumerate}[label=\arabic*.]
  \setcounter{enumi}{15}
  \item \verb|message_ref| : upstream messages/logs (ISO20022, ISO8583, tx hashes).
  \item \verb|purpose_code| : business purpose.
  \item \verb|expiry_time| : for encumbrances with timeouts (HTLCs, auth holds, etc.).
  \item \verb|notes| : free text.
\end{enumerate}

%%%%%%%%%%%%%%%%%%%%%%%%%%%%%%%%%%%%%%%%%%%%%%%%%%%%%%%%%%%%
\section{Pending Events}
%%%%%%%%%%%%%%%%%%%%%%%%%%%%%%%%%%%%%%%%%%%%%%%%%%%%%%%%%%%%

Introduce Pending-BSTE (PBSTE) as an operational extension:

\begin{itemize}
  \item Structural fields mirror BSTE.
  \item Additional fields:
    \begin{itemize}[noitemsep]
      \item \verb|t_proposed|,
      \item \verb|status| $\in$ \{pending, accepted, rejected, cancelled\}.
    \end{itemize}
\end{itemize}

PBSTEs do not affect:
\begin{itemize}[noitemsep]
  \item supply $S(t)$,
  \item encumbrance $K(t)$,
  \item free balances in the canonical historical accounting.
\end{itemize}

They do affect projected liquidity:
\[
  F_{\mathrm{projected}}(t)
  = F_{\mathrm{posted}}(t)
    - \sum_{\text{pending outflows}} \mathrm{amount} .
\]

%%%%%%%%%%%%%%%%%%%%%%%%%%%%%%%%%%%%%%%%%%%%%%%%%%%%%%%%%%%%
\section{Global Invariants and Bridging Semantics}
%%%%%%%%%%%%%%%%%%%%%%%%%%%%%%%%%%%%%%%%%%%%%%%%%%%%%%%%%%%%

\subsection{Per-Ledger Supply and Encumbrance}

For each (\texttt{ledger\_id}, \texttt{asset\_code}):

\begin{itemize}
  \item Supply:
    \[
      S(t) = S(t_0) + \sum_{\text{events }\le t} \mathrm{supply\_delta} .
    \]
  \item Encumbered quantity:
    \[
      K(t) = \sum_{\text{accounts, encumbered buckets}} \mathrm{balance}(t) .
    \]
  \item Free balance per account:
    \[
      \mathrm{FreeBalance}(t) \ge 0.
    \]
\end{itemize}

\subsection{Primitive-Level Constraints}

\begin{itemize}[noitemsep]
  \item \textbf{P1 OWNERSHIP\_TRANSFER}: \verb|supply_delta = 0|.
  \item \textbf{P2 ENCUMBRANCE\_ADJUST}: \verb|supply_delta = 0| and
        $\mathrm{econ\_owner}(\mathrm{src}) = \mathrm{econ\_owner}(\mathrm{dst})$.
  \item \textbf{P3 SUPPLY\_ADJUST}: exactly one endpoint is external.
\end{itemize}

\subsection{Atomic Sets}

For each \verb|link_id|, there is an entry in \texttt{atomic\_sets} describing the intended pattern and mechanism.

Invariants include:

\begin{itemize}[noitemsep]
  \item For PvP: no leg should settle in isolation (interpretation depends on mechanism).
  \item For DvP: cash and security legs are coupled.
  \item For HTLCs: encumbrances must be released by either success or timeout.
\end{itemize}

\subsection{Bridging Invariants}

For lock-and-mint bridges:

\begin{itemize}
  \item Backing ledger uses \verb|bridge_lock| buckets for locked units.
  \item Wrapped asset ledger mints new tokens via \textbf{SUPPLY\_ADJUST}.
\end{itemize}

A simple invariant for one-to-one lock-mint:

\[
  S_{\mathrm{wrapped}}(t) \le K_{\mathrm{backing, bridge\_lock}}(t)
\]
(equality up to fees/slippage).

%%%%%%%%%%%%%%%%%%%%%%%%%%%%%%%%%%%%%%%%%%%%%%%%%%%%%%%%%%%%
\section{Digital Liquidity Stack (Core and Extended)}
%%%%%%%%%%%%%%%%%%%%%%%%%%%%%%%%%%%%%%%%%%%%%%%%%%%%%%%%%%%%

\subsection{Core Dimensions}

These are the 10 core fields used for classification tables.

\begin{enumerate}[label=\textbf{D\arabic*}.]
  \item \textbf{asset\_nature}:
    cb\_reserve, cb\_cash, commercial\_deposit, e\_money, stablecoin\_fiat,
    stablecoin\_crypto, token\_native, security\_cash, other.
  \item \textbf{legal\_form}:
    balance\_sheet\_claim, trust\_unit, fund\_share, bearer\_instrument,
    synthetic\_derivative, cb\_direct.
  \item \textbf{representation\_model}:
    native\_account, native\_token, wrapped\_mirror, synthetic.
  \item \textbf{ledger\_tech}:
    cb\_rtgs, bank\_core, ccp\_cash\_ledger, cls\_pvp,
    dlt\_public, dlt\_permissioned, scheme\_internal, channel\_state, other.
  \item \textbf{account\_model}:
    account\_balances, utxo, smart\_contract\_state, hybrid.
  \item \textbf{scheme\_type}:
    rtgs\_operator, instant\_payments, ach\_dns, card\_scheme,
    correspondent\_network, dex\_protocol, bridge\_protocol,
    mobile\_money\_scheme, other.
  \item \textbf{access\_model}:
    direct, indirect, retail, wholesale, permissionless, permissioned.
  \item \textbf{clearing\_mechanism}:
    none\_gross, bilateral\_net, multilateral\_net, queue\_lsm,
    continuous\_net, offchain\_channels.
  \item \textbf{settlement\_mode}:
    gross, bilateral\_net\_batch, multilateral\_net\_batch,
    hybrid\_queue\_lsm, onchain\_gross, onchain\_batch.
  \item \textbf{finality\_kind} and \textbf{consistency\_model}:
    deterministic / deterministic\_deferred / probabilistic vs
    eventual\_reconciliation / consensus\_atomic / hybrid.
\end{enumerate}

\subsection{Extended Dimensions}

Extended fields that can be used in deeper analysis or thesis-only tables:

\begin{itemize}
  \item custody\_model, authorisation\_model, freeze\_authority.
  \item credit\_sources, encumbrance\_eligibility, allows\_negative\_balances.
  \item optimisation\_style, optimisation\_scope.
  \item messaging\_standards, addressing\_scheme, id\_scheme.
  \item privacy\_model, replay\_guard.
  \item governance\_model, legal\_finality\_basis.
\end{itemize}

%%%%%%%%%%%%%%%%%%%%%%%%%%%%%%%%%%%%%%%%%%%%%%%%%%%%%%%%%%%%
\section{Tokenised vs Non-Tokenised vs Hybrid}
%%%%%%%%%%%%%%%%%%%%%%%%%%%%%%%%%%%%%%%%%%%%%%%%%%%%%%%%%%%%

\subsection{Non-Tokenised Systems}

Common characteristics:

\begin{itemize}[noitemsep]
  \item representation\_model = native\_account,
  \item ledger\_tech = cb\_rtgs / bank\_core / scheme\_internal,
  \item consistency\_model = eventual\_reconciliation,
  \item execution is message-driven, core state is opaque during processing.
\end{itemize}

\subsection{Tokenised Systems}

Common characteristics:

\begin{itemize}[noitemsep]
  \item representation\_model $\in$ \{native\_token, wrapped\_mirror\},
  \item ledger\_tech = dlt\_public or dlt\_permissioned,
  \item consistency\_model = consensus\_atomic,
  \item programmability: smart contracts see state and update in the same transaction.
\end{itemize}

\subsection{Hybrid Systems}

Examples:

\begin{itemize}[noitemsep]
  \item Tokenised deposits that sit atop bank cores but expose a token interface.
  \item RLN / unified-ledger designs with central-bank and commercial-bank tiers.
  \item Card schemes or PSPs that mirror balances onto a DLT sub-ledger.
\end{itemize}

Discussion will emphasise:
\begin{itemize}[noitemsep]
  \item how hybrid systems occupy intermediate coordinates in the stack,
  \item distinct failure modes and liquidity behaviours.
\end{itemize}

%%%%%%%%%%%%%%%%%%%%%%%%%%%%%%%%%%%%%%%%%%%%%%%%%%%%%%%%%%%%
\section{Taxonomy of Liquidity Mechanisms}
%%%%%%%%%%%%%%%%%%%%%%%%%%%%%%%%%%%%%%%%%%%%%%%%%%%%%%%%%%%%}

This section classifies mechanisms as compositions of BSTEs at specific stack coordinates.

\subsection{Credit and Funding}

\begin{itemize}[noitemsep]
  \item Intraday central-bank credit (RTGS).
  \item Overdrafts and bilateral credit lines.
  \item Repo-based liquidity provision.
\end{itemize}

\subsection{Encumbrances and Collateral}

\begin{itemize}[noitemsep]
  \item Holds (card, instant payments).
  \item CCP margin and haircuts.
  \item Collateralisation at central bank / CCP / DLT-based vaults.
\end{itemize}

\subsection{Clearing and Netting}

\begin{itemize}[noitemsep]
  \item Bilateral and multilateral netting.
  \item LSM / gridlock resolution.
  \item DNS transfer cycles and settlement windows.
\end{itemize}

\subsection{Channels and Off-Chain Mechanisms}

\begin{itemize}[noitemsep]
  \item Payment channels (Lightning and analogues).
  \item State channels, rollups with periodic settlement.
\end{itemize}

\subsection{PvP, DvP, PoP, and Bridges}

\begin{itemize}[noitemsep]
  \item PvP FX legs across RTGS or DLTs.
  \item DvP securities settlement vs cash.
  \item PoP patterns where obligations are offset.
  \item Bridge designs: custodial lock-mint, burn-and-mint, synthetic representations.
\end{itemize}

%%%%%%%%%%%%%%%%%%%%%%%%%%%%%%%%%%%%%%%%%%%%%%%%%%%%%%%%%%%%
\section{Representative Case Studies}
%%%%%%%%%%%%%%%%%%%%%%%%%%%%%%%%%%%%%%%%%%%%%%%%%%%%%%%%%%%%

This section will contain 3--5 in-depth examples, each with:

\begin{itemize}[noitemsep]
  \item stack coordinates,
  \item BSTE sequences for typical flows,
  \item encumbrance and credit implications.
\end{itemize}

Candidate case studies:

\begin{enumerate}
  \item \textbf{Central-bank RTGS with LSM}.  
        Show queued payments, LSM runs, BSTE-level effect.
  \item \textbf{Card scheme + DNS}.  
        Auth holds, clearing batches, RTGS settlement, chargebacks.
  \item \textbf{Instant retail system (e.g., NPP-like)}.  
        Real-time credits with holds and CB RTGS backing.
  \item \textbf{DeFi AMM swap + cross-chain bridge}.  
        Multi-leg on-chain bundles, HTLCs, finality and latency.
  \item \textbf{Tokenised deposit / unified ledger}.  
        Hybrid consistency, legal form, governance.
\end{enumerate}

%%%%%%%%%%%%%%%%%%%%%%%%%%%%%%%%%%%%%%%%%%%%%%%%%%%%%%%%%%%%
\section{Design Space and Discussion}
%%%%%%%%%%%%%%%%%%%%%%%%%%%%%%%%%%%%%%%%%%%%%%%%%%%%%%%%%%%%}

Here the chapter synthesises:

\begin{itemize}
  \item How different systems cluster in the 10-dimensional core space.
  \item Trade-offs:
    \begin{itemize}[noitemsep]
      \item pre-funded vs credit-backed,
      \item gross vs net settlement,
      \item centralised vs consensus-based finality,
      \item transparency vs privacy,
      \item simplicity vs programmability.
    \end{itemize}
  \item Failure modes:
    \begin{itemize}[noitemsep]
      \item reconciliation drift,
      \item reorg risk,
      \item stuck encumbrances (e.g., HTLC timeouts, unresolved holds),
      \item bridge misconfigurations and backing failures.
    \end{itemize}
\end{itemize}

This section also positions emerging designs (RLN, unified ledgers, tokenised collateral networks) in the space.

%%%%%%%%%%%%%%%%%%%%%%%%%%%%%%%%%%%%%%%%%%%%%%%%%%%%%%%%%%%
\section{Subsequent Papers Overview}
%%%%%%%%%%%%%%%%%%%%%%%%%%%%%%%%%%%%%%%%%%%%%%%%%%%%%%%%%%%%}



\subsection{Stylised Facts of Tokenised Real World Asset Markets (Paper 2)}

Explain how:

\begin{itemize}[noitemsep]
  \item BSTE encoding allows uniform extraction of transaction and settlement data.
  \item Stack coordinates inform:
    \begin{itemize}[noitemsep]
      \item which systems are comparable,
      \item where latencies and failures originate.
    \end{itemize}
  \item Empirical stylised facts can then be tied explicitly to plumbing choices.
\end{itemize}

\subsection{Collateral as Settlement Asset (Paper 3)}

Show how:

\begin{itemize}[noitemsep]
  \item legal\_form, governance\_model, and encumbrance semantics in the stack
        become central to designing repo-native money and collateral tokens.
  \item invariants from the SoK constrain safe designs for yield-bearing settlement assets.
\end{itemize}

\subsection{Agentic Liquidity (Paper 4)}

Outline how the BSTE + stack model supports:

\begin{itemize}[noitemsep]
  \item Formal graph representations of multi-ledger liquidity.
  \item Problem statements of the form: \\
        ``Given required OWNERSHIP\_TRANSFER BSTEs, 
        find ENCUMBRANCE\_ADJUST and routing decisions that minimise
        a cost functional such as $\int K(t)\, dt$ subject to constraints.''
  \item Integration of LSM/queueing, MILP/MPC, RL agents.
\end{itemize}



% %%%%%%%%%%%%%%%%%%%%%%%%%%%%%%%%%%%%%%%%%%%%%%%%%%%%%%%%%%%%
% \section{Writing Plan: Article vs Thesis Chapter}
% %%%%%%%%%%%%%%%%%%%%%%%%%%%%%%%%%%%%%%%%%%%%%%%%%%%%%%%%%%%%}

% \subsection{Journal Article (Condensed Version)}

% For a 15--20 page SoK article (e.g., JSys):

% \begin{itemize}
%   \item Include:
%     \begin{itemize}[noitemsep]
%       \item Introduction \& definition of digital liquidity.
%       \item BSTE model and core invariants.
%       \item 10-dimension core stack.
%       \item Mechanism taxonomy (compressed).
%       \item 3--4 representative case studies.
%       \item Design space \& research agenda.
%     \end{itemize}
%   \item Keep extended attributes (governance, privacy, etc.) light but present.
% \end{itemize}

% \subsection{Thesis Chapter (Extended Version)}

% The thesis chapter can include:

% \begin{itemize}
%   \item Richer related-work section.
%   \item Full extended stack tables (governance, legal finality, privacy).
%   \item More detailed case studies and diagrams.
%   \item Methodological notes on how systems were selected \& classified.
%   \item Explicit cross-references to subsequent thesis chapters.
% \end{itemize}

% %%%%%%%%%%%%%%%%%%%%%%%%%%%%%%%%%%%%%%%%%%%%%%%%%%%%%%%%%%%%
% \section*{Status}
% %%%%%%%%%%%%%%%%%%%%%%%%%%%%%%%%%%%%%%%%%%%%%%%%%%%%%%%%%%%%}

% This document is a \emph{planning outline}.  
% Next steps:
% \begin{itemize}[noitemsep]
%   \item Turn each section into full prose.
%   \item Add references and citations.
%   \item Design classification tables and small diagrams based on this skeleton.
% \end{itemize}

\end{document}
