\documentclass[11pt,aspectratio=169]{beamer}

% Use a clean theme that works well with TikZ
\usetheme{Dresden}
\usecolortheme{crane}

% Customize the headline to remove section names but keep the yellow bar
\setbeamertemplate{headline}
{
  \leavevmode%
  \hbox{%
  \begin{beamercolorbox}[wd=\paperwidth,ht=2.25ex,dp=1.125ex]{palette tertiary}%
  \end{beamercolorbox}%
  }%
}

% Override crane's yellow frametitle with a pale blue
\definecolor{paleblue}{HTML}{e0eaff}
\setbeamercolor{frametitle}{fg=black,bg=paleblue}

% Override the yellow top bar
\definecolor{yellowbar}{HTML}{ffbf00}
\setbeamercolor{palette tertiary}{fg=black,bg=yellowbar}

% Use serif fonts for academic look
\usefonttheme{serif}
\usefonttheme{professionalfonts}

% Font packages
\usepackage[T1]{fontenc}
\usepackage{libertine}  % Beautiful serif font
\usepackage[libertine]{newtxmath}  % Matching math font

% Essential TikZ libraries
\usepackage{tikz}
\usetikzlibrary{
    arrows.meta,
    backgrounds,
    calc,
    decorations.pathreplacing,
    decorations.pathmorphing,
    decorations.text,
    fadings,
    fit,
    graphs,
    matrix,
    mindmap,
    patterns,
    positioning,
    shadows,
    shapes.geometric,
    shapes.symbols,
    trees,
    3d,
    automata
}

% PGFPlots for data visualization
\usepackage{pgfplots}
\pgfplotsset{compat=1.18}
\usepgfplotslibrary{statistics}
\usepgfplotslibrary{colorbrewer}

% Colors optimized for TikZ
\definecolor{tikzred}{RGB}{255,58,71}
\definecolor{tikzblue}{RGB}{0,103,184}
\definecolor{tikzgreen}{RGB}{0,146,69}
\definecolor{tikzorange}{RGB}{255,140,0}
\definecolor{tikzpurple}{RGB}{103,78,167}
\definecolor{tikzgray}{RGB}{100,100,100}
\definecolor{lightgray}{RGB}{240,240,240}

% Custom TikZ styles
\tikzset{
    % Box styles
    concept box/.style={
        draw=tikzblue,
        fill=tikzblue!10,
        thick,
        rectangle,
        rounded corners,
        minimum height=1cm,
        minimum width=2.5cm,
        align=center
    },
    data box/.style={
        draw=tikzgreen,
        fill=tikzgreen!10,
        thick,
        rectangle,
        minimum height=0.8cm,
        align=center
    },
    process box/.style={
        draw=tikzorange,
        fill=tikzorange!10,
        thick,
        ellipse,
        minimum height=0.8cm,
        align=center
    },
    % Arrow styles
    flow arrow/.style={
        ->,
        thick,
        >=Stealth,
        tikzblue
    },
    data flow/.style={
        ->,
        thick,
        >=Latex,
        tikzgreen,
        dashed
    },
    % Graph node styles
    main node/.style={
        circle,
        draw=tikzblue,
        fill=tikzblue!10,
        thick,
        minimum size=0.8cm,
        font=\footnotesize\bfseries
    },
    % Highlighting
    highlight/.style={
        draw=tikzred,
        fill=tikzred!10,
        ultra thick
    }
}

% Additional packages
\usepackage{booktabs}
\usepackage{animate}
\usepackage{datetime}
\usepackage{colortbl}  % For row coloring in tables
\usepackage{amssymb}   % For symbols like \multimap

% Counter for subsections within frames
\newcounter{framesubsection}[section]

% Custom invisible block style for subsections with auto-numbering
\newenvironment{autosubsection}[1]{%
    \refstepcounter{framesubsection}%
    \vspace{0.6em}%
    \noindent{\normalsize \textbf{\thesection.\theframesubsection\quad #1}}\par%
    \vspace{0.2em}%
    \small%
    \leftskip=1.5em%
}{\par\leftskip=0em\normalsize}

\newcounter{framesubsubsection}[framesubsection]

% Custom invisible block style for subsubsections with auto-numbering
\newenvironment{autosubsubsection}[1]{%
    \refstepcounter{framesubsubsection}%
    \vspace{0.1em}%
    \hspace{1em}{\small \thesection.\theframesubsection.\theframesubsubsection\quad #1}\par%
}{}

\newcounter{framesubsubsubsection}[framesubsubsection]

% Custom invisible block style for subsubsubsections with auto-numbering
\newenvironment{autosubsubsubsection}[1]{%
    \refstepcounter{framesubsubsubsection}%
    \vspace{0.05em}%
    \hspace{2em}{\footnotesize \thesection.\theframesubsection.\theframesubsubsection.\theframesubsubsubsection\quad #1}\par%
}{}

% Hide subsections from table of contents
\setcounter{tocdepth}{1}

% Set beamer specific options
\setbeamertemplate{navigation symbols}{}

% Custom footline with yellow bar at bottom
\setbeamertemplate{footline}{
  \leavevmode%
  \hbox{%
  \begin{beamercolorbox}[wd=\paperwidth,ht=2.25ex,dp=1.125ex]{palette tertiary}%
    \hfill\usebeamerfont{footline}\insertframenumber{} / \inserttotalframenumber\hspace*{2ex}
  \end{beamercolorbox}%
  }%
}

% Title
\title{Liquidity \& Token Interchange}
\subtitle{Problems in Computational Settlement}
\author{Gerard Kelly}
\institute{%
  University of New South Wales\\
  School of Computer Science \& Engineering
  \and
  Digital Finance Cooperative Research Centre
}
\date{\monthname\ \the\year}

\begin{document}

% Title slide with TikZ decoration
\begingroup
\setbeamertemplate{headline}{} % Remove headline for title slide
\setbeamertemplate{footline}{} % Remove footline for title slide
\setbeamertemplate{navigation symbols}{} % Remove navigation symbols
\setbeamertemplate{background canvas}{
    % Plain white background - no decorations
}
\begin{frame}[plain,noframenumbering]
    \vspace{1cm}
    \begin{center}
        {\Huge \textbf{\inserttitle}}\\[0.5em]
        {\Large \insertsubtitle}\\[1.5em]
        {\large \insertauthor}\\[0.5em]
        {\insertinstitute}\\[1em]
        {\insertdate}
    \end{center}
\end{frame}
\endgroup

% Table of Contents
\begin{frame}{Outline}
    \tableofcontents
\end{frame}

% Section 1: Introduction 
\section{1. Introduction}

\begin{frame}{1. Introduction}

    \begin{autosubsection}{The Problem of Settlement}
        Introduction to payments settlement as a computational problem; general motivation.
    \end{autosubsection}

    \begin{autosubsection}{Liquidity \& Tokenisation}
        Definitions of liquidity, tokens, interchange, money, and the contemporary industry context.
    \end{autosubsection}

    \begin{autosubsection}{Computer Science \& Digital Finance}
        Overview of formal methods in Computer Science with application to finance (preview).
    \end{autosubsection}

    \begin{autosubsection}{Research Questions}
        Scope of project and enumeration of primary research questions.
    \end{autosubsection}

    \begin{autosubsection}{Structure \& Contributions}
        Outline of subsequent sections; methodologies, publications, and key research outcomes. 
    \end{autosubsection}

\end{frame}

% Section 2: Liquidity Automata
\section{2. Liquidity Automata}

\begin{frame}{2. Liquidity Automata}

\textbf{Liquidity Automata: A Computational Hierarchy of Money Forms}
\\[1em]
\small
This chapter (and corresponding paper) will propose a correspondence between the Chomsky heirarchy of computational complexity and distinctive forms of money and settlement that have emerged over history. The proposed hierarchy offers a theoretical framework for the relatioship between liqudity and technology that identifies four key innovations in its evolution; a process beginning with the adoption of physical currency and culminating with computationally universal programmable tokens.\\
[1em]
The chapter will include a review of the canonical literature relating to computational complexity and monetary theory. It will argue that the undecidability of tokenised monetary systems introduces novel forms of risk and compels the management of settlement operations through the deployment of \textbf{keepers} and other types of autonomous agents. Concepts introduced in this chapter will provide a motivation and theoretical foundation for subsequent chapters.

\end{frame}

% % Second slide for Liquidity Automata - Chomsky Hierarchy Diagram
% \begin{frame}{2. Liquidity Automata}
%     \begin{center}
%     \begin{tikzpicture}[scale=0.8, transform shape]
%         % Define colors for each level
%         \definecolor{type0color}{RGB}{200,230,255}
%         \definecolor{type1color}{RGB}{180,210,240}
%         \definecolor{type2color}{RGB}{160,190,225}
%         \definecolor{type3color}{RGB}{140,170,210}

%         % Left side: Nested hierarchy of languages
%         \node[text width=5cm, align=center] at (-4, 5.5) {\textbf{Language Classes}};

%         % Type 0 - Recursively Enumerable
%         \draw[very thick, fill=type0color] (-7,-1.8) rectangle (-1,3.9);
%         \node[align=center] at (-4,3.3) {Type 0: Recursively Enumerable};

%         % Type 1 - Context-Sensitive
%         \draw[very thick, fill=type1color] (-6.5,-1.7) rectangle (-1.5,2.4);
%         \node[align=center] at (-4,1.8) {Type 1: Context-Sensitive};

%         % Type 2 - Context-Free
%         \draw[very thick, fill=type2color] (-6,-1.4) rectangle (-2,0.9);
%         \node[align=center] at (-4,0.3) {Type 2: Context-Free};

%         % Type 3 - Regular
%         \draw[very thick, fill=type3color] (-5,-1.3) rectangle (-2.5,-0.7);
%         \node[align=center] at (-4,-1.0) {Type 3: Regular};

%         % Right side: Corresponding automata
%         \node[text width=5cm, align=center] at (4, 5.5) {\textbf{Automata Types}};

%         % Turing Machine
%         \node[concept box, fill=type0color, minimum width=4cm, minimum height=1.2cm] at (4,3.3) {Turing Machine\\(Unbounded tape)};

%         % Linear Bounded Automaton
%         \node[concept box, fill=type1color, minimum width=4cm, minimum height=1.2cm] at (4,1.8) {Linear Bounded\\Automaton};

%         % Pushdown Automaton
%         \node[concept box, fill=type2color, minimum width=4cm, minimum height=1.2cm] at (4,0.3) {Pushdown\\Automaton};

%         % Finite State Automaton
%         \node[concept box, fill=type3color, minimum width=4cm, minimum height=1.2cm] at (4,-1.0) {Finite State\\Automaton};

%         % Arrows connecting languages to automata - all horizontal
%         \draw[flow arrow, line width=1.5pt] (-1,3.3) -- (1.8,3.3);
%         \draw[flow arrow, line width=1.5pt] (-1.5,1.8) -- (1.8,1.8);
%         \draw[flow arrow, line width=1.5pt] (-2,0.3) -- (1.8,0.3);
%         \draw[flow arrow, line width=1.5pt] (-2.5,-1.0) -- (1.8,-1.0);

%     \end{tikzpicture}
%     \end{center}
% \end{frame}

% Section 3: Liquidity Logics
\section{3. Liquidity Logics}

\begin{frame}{3. Liquidity Logics}

\textbf{Liquidity Logics: Settlement as Computational Resource Allocation}
\\[1em]
\small
Surveying the broader field of Computer Science, this chapter will adopt formal approaches to computational resource constraints to the problem of settlement. A framework for \textbf{Balance State Transition Events} that combines elements from four main fields of theory will aim to model settlement systems that are linearly-constrained, flow-optimized, strategically robust, and privacy-preserving. This chapter will constitute the primary literature review material of the thesis.\\
[1em]
\textbf{Linear Logic and Petri Nets} will provide a foundation for resource consumption and conservation, addressing risks of insolvency and rehypothecation. \textbf{Network Calculus and Queueing Theory} provide tools for analysing flow and liveness properties. \textbf{Consensus Mechanisms and Strategic Game Theory} will survey topics including BFT and MEV - exploring finality of settlement and alignment of participant incentives. \textbf{Zero-Knowledge Proofs and Privacy} will examine how settlement can be both private and verifiable, comparing alternative approaches and their respective computational costs.
\end{frame}

% Section 4: Liquidity Networks
\section{4. Liquidity Networks}

\begin{frame}{4. Liquidity Networks}

\textbf{Liquidity Networks: Stylised Facts of Early Tokenised Finance}
\\[1em]
\small
This empirical chapter will provide a comprehensive analysis of on-chain data from the Ethereum ecosystem (L1 and L2) and other significant networks. The chapter will examine patterns of stablecoin and token usage, major liquidity pools and flows, and the specific token standards (e.g., ERC-4626) driving institutional adoption.\\
[1em]
Through network analysis and statistical methods, the research will identify bot networks, quantify wash trading, and estimate the proportion of actual human users in DeFi protocols versus algorithmic actors. Key metrics will include: the percentage of trades representing arbitrage versus organic trading, concentration of liquidity provision, cross-chain flow patterns, and the emergence of institutional-grade infrastructure.\\
[1em]
By applying network theory to transaction graphs and liquidity flows, this chapter aims to establish empirical baselines for understanding tokenised financial networks - distinguishing between genuine adoption patterns and artificial activity, while documenting the evolution of on-chain financial primitives from experimental protocols to production-grade systems.
\end{frame}

% Section 5: Liquidity Scheme Design
\section{5. Liquidity Schemes}

\begin{frame}{5. Liquidity Schemes}
    % Content to be added
\end{frame}

% Section 6: Liquidity Agents
\section{6. Liquidity Agents}

\begin{frame}{6. Liquidity Agents}
    % Content to be added
\end{frame}

% Section 7: Discussion and Conclusion
\section{7. Conclusion}

\begin{frame}{7. Conclusion}
    Review and discussion of results; directions for future research.
\end{frame}

\end{document}