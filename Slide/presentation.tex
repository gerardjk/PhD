\documentclass[11pt,aspectratio=169]{beamer}

% Use a clean theme that works well with TikZ
\usetheme{Dresden}
\usecolortheme{crane}

% Customize the headline to remove section names but keep the yellow bar
\setbeamertemplate{headline}
{
  \leavevmode%
  \hbox{%
  \begin{beamercolorbox}[wd=\paperwidth,ht=2.25ex,dp=1.125ex]{palette tertiary}%
  \end{beamercolorbox}%
  }%
}

% Override crane's yellow frametitle with a pale blue
\definecolor{paleblue}{HTML}{e0eaff}
\setbeamercolor{frametitle}{fg=black,bg=paleblue}

% Override the yellow top bar
\definecolor{yellowbar}{HTML}{ffbf00}
\setbeamercolor{palette tertiary}{fg=black,bg=yellowbar}

% Use serif fonts for academic look
\usefonttheme{serif}
\usefonttheme{professionalfonts}

% Font packages
\usepackage[T1]{fontenc}
\usepackage{libertine}  % Beautiful serif font
\usepackage[libertine]{newtxmath}  % Matching math font

% Essential TikZ libraries
\usepackage{tikz}
\usetikzlibrary{
    arrows.meta,
    backgrounds,
    calc,
    decorations.pathreplacing,
    decorations.pathmorphing,
    decorations.text,
    fadings,
    fit,
    graphs,
    matrix,
    mindmap,
    patterns,
    positioning,
    shadows,
    shapes.geometric,
    shapes.symbols,
    trees,
    3d,
    automata
}

% PGFPlots for data visualization
\usepackage{pgfplots}
\pgfplotsset{compat=1.18}
\usepgfplotslibrary{statistics}
\usepgfplotslibrary{colorbrewer}

% Colors optimized for TikZ
\definecolor{tikzred}{RGB}{255,58,71}
\definecolor{tikzblue}{RGB}{0,103,184}
\definecolor{tikzgreen}{RGB}{0,146,69}
\definecolor{tikzorange}{RGB}{255,140,0}
\definecolor{tikzpurple}{RGB}{103,78,167}
\definecolor{tikzgray}{RGB}{100,100,100}
\definecolor{lightgray}{RGB}{240,240,240}

% Custom TikZ styles
\tikzset{
    % Box styles
    concept box/.style={
        draw=tikzblue,
        fill=tikzblue!10,
        thick,
        rectangle,
        rounded corners,
        minimum height=1cm,
        minimum width=2.5cm,
        align=center
    },
    data box/.style={
        draw=tikzgreen,
        fill=tikzgreen!10,
        thick,
        rectangle,
        minimum height=0.8cm,
        align=center
    },
    process box/.style={
        draw=tikzorange,
        fill=tikzorange!10,
        thick,
        ellipse,
        minimum height=0.8cm,
        align=center
    },
    % Arrow styles
    flow arrow/.style={
        ->,
        thick,
        >=Stealth,
        tikzblue
    },
    data flow/.style={
        ->,
        thick,
        >=Latex,
        tikzgreen,
        dashed
    },
    % Graph node styles
    main node/.style={
        circle,
        draw=tikzblue,
        fill=tikzblue!10,
        thick,
        minimum size=0.8cm,
        font=\footnotesize\bfseries
    },
    % Highlighting
    highlight/.style={
        draw=tikzred,
        fill=tikzred!10,
        ultra thick
    }
}

% Additional packages
\usepackage{booktabs}
\usepackage{animate}
\usepackage{datetime}

% Counter for subsections within frames
\newcounter{framesubsection}[section]

% Custom invisible block style for subsections with auto-numbering
\newenvironment{autosubsection}[1]{%
    \refstepcounter{framesubsection}%
    \vspace{0.6em}%
    \noindent{\normalsize \textbf{\thesection.\theframesubsection\quad #1}}\par%
    \vspace{0.2em}%
    \small%
    \leftskip=1.5em%
}{\par\leftskip=0em\normalsize}

\newcounter{framesubsubsection}[framesubsection]

% Custom invisible block style for subsubsections with auto-numbering
\newenvironment{autosubsubsection}[1]{%
    \refstepcounter{framesubsubsection}%
    \vspace{0.1em}%
    \hspace{1em}{\small \thesection.\theframesubsection.\theframesubsubsection\quad #1}\par%
}{}

\newcounter{framesubsubsubsection}[framesubsubsection]

% Custom invisible block style for subsubsubsections with auto-numbering
\newenvironment{autosubsubsubsection}[1]{%
    \refstepcounter{framesubsubsubsection}%
    \vspace{0.05em}%
    \hspace{2em}{\footnotesize \thesection.\theframesubsection.\theframesubsubsection.\theframesubsubsubsection\quad #1}\par%
}{}

% Hide subsections from table of contents
\setcounter{tocdepth}{1}

% Set beamer specific options
\setbeamertemplate{navigation symbols}{}

% Custom footline with yellow bar at bottom
\setbeamertemplate{footline}{
  \leavevmode%
  \hbox{%
  \begin{beamercolorbox}[wd=\paperwidth,ht=2.25ex,dp=1.125ex]{palette tertiary}%
    \hfill\usebeamerfont{footline}\insertframenumber{} / \inserttotalframenumber\hspace*{2ex}
  \end{beamercolorbox}%
  }%
}

% Title
\title{Liquidity \& Token Interchange}
\subtitle{Problems in Computational Settlement}
\author{Gerard Kelly}
\institute{University of New South Wales\\School of Computer Science \& Engineering}
\date{\monthname\ \the\year}

\begin{document}

% Title slide with TikZ decoration
\begingroup
\setbeamertemplate{headline}{} % Remove headline for title slide
\setbeamertemplate{footline}{} % Remove footline for title slide
\setbeamertemplate{navigation symbols}{} % Remove navigation symbols
\setbeamertemplate{background canvas}{
    % Plain white background - no decorations
}
\begin{frame}[plain,noframenumbering]
    \vspace{1cm}
    \begin{center}
        {\Huge \textbf{\inserttitle}}\\[0.5em]
        {\Large \insertsubtitle}\\[1.5em]
        {\large \insertauthor}\\[0.5em]
        {\insertinstitute}\\[1em]
        {\insertdate}
    \end{center}
\end{frame}
\endgroup

% Table of Contents
\begin{frame}{Outline}
    \tableofcontents
\end{frame}

% Section 1: Introduction 
\section{1. Introduction}

\begin{frame}{1. Introduction}

    \begin{autosubsection}{The Hard Problem of Settlement}
        Introduction to payments settlement as a computational problem; historical background. 
    \end{autosubsection}

    \begin{autosubsection}{Liquidity \& Tokenisation}
        Definitions of liquidity, tokens, interchange, and the contemporary industry context.
    \end{autosubsection}

    \begin{autosubsection}{Computer Science \& Digital Finance}
        Overview of formal methods in Computer Science with application to finance.
    \end{autosubsection}

    \begin{autosubsection}{Research Questions}
        Enumeration of primary research questions addressed by this project.
    \end{autosubsection}

    \begin{autosubsection}{Structure \& Contributions}
        Outline of subsequent sections; methodologies, publications, and key research outcomes. 
    \end{autosubsection}

\end{frame}

\begin{frame}{1. Introduction}

\textbf{1.4 Research Questions}



\end{frame}

% Section 2: Liquidity Automata
\section{2. Liquidity Automata}

\begin{frame}{2. Liquidity Automata}

\textbf{Liquidity Automata: A Computational Hierarchy of Money Forms}
\\[1em]
\small
This chapter (and corresponding paper) will propose a correspondence between the Chomsky heirarchy of computational complexity and distinctive forms of money and settlement that have emerged over time. \\
[1em]
The proposed hierarchy provides a theoretical framework for the relatioship between liqudity and technology that identifies four key innovations in its evolution - a process culminating with digital tokenisation and the arrival of computational universality.\\
[1em]
The chapter will include a review of the canonical literature relating to computation and economics. It will argue that the complexity of tokenised monetary systems introduce an imperative to manage settlement risks through the development of autonomous agents. Although abstract, concepts introduced in this chapter will provide theoretical foundations for subsequent chapters.

\end{frame}

% Section 3: Liquidity Logics
\section{3. Liquidity Logics}

\begin{frame}{3. Liquidity Logics}
    % Content to be added
\end{frame}

% Section 4: Liquidity Networks
\section{4. Liquidity Networks}

\begin{frame}{4. Liquidity Networks}
    % Content to be added
\end{frame}

% Section 5: Liquidity Scheme Design
\section{5. Liquidity Schemes}

\begin{frame}{5. Liquidity Schemes}
    % Content to be added
\end{frame}

% Section 6: Liquidity Agents
\section{6. Liquidity Agents}

\begin{frame}{6. Liquidity Agents}
    % Content to be added
\end{frame}

% Section 7: Discussion and Conclusion
\section{7. Conclusion}

\begin{frame}{7. Conclusion}
    % Content to be added
\end{frame}

\end{document}